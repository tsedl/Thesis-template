%=== Chapter Four ===
\chapter{Conflict Points at Intersections and Adaptive Paths}
\section{Introduction}
The capacity and safety of an intersection can be considered in terms of the number and type of conflict points which it contains. When every entrance to the intersection is connected to every exit by a smooth arc, a conflict point exists wherever two arcs intersect. They can be classified as head-on, following, crossing, diverging or merging. In many studies these points are assumed to be fixed, and only speed of the vehicles which controls the rate of progress along the arcs is varied to minimise delay. 

With the development of path planning around obstacles on-the-fly, it becomes possible to consider what arrangement of conflict points is best given the traffic at a particular instant in the near future. 

\section{Literature Review} 
Studies on the theoretical capacity of signalized intersections and roundabouts with an equivalent footprint indicate that in most cases, if there are few approach lanes small roundabouts will tend to have higher capacity. If there are many approach lanes signals tend to be more effective, unless the traffic on different approaches is extremely unequal \cite{Jian-an2001}. 

A systematic procedure computing the conflict points in an intersection is given in \cite{Lu2013}. Roundabouts are dominated by a large number of merging and diverging conflicts, with few of the more dangerous crossing and head-on conflicts.

Intersection control often addresses crossing conflicts by separating vehicles in time, while they all take the shortest path straight through the intersection in the same way as if it was signal controlled. There are a wide range of optimal and heuristic approaches for design, both decentralized and centralized, a good review  is given in \cite{Malikopoulos2018}. Many studies have looked at how to incorporate a proportion of human controlled vehicles which are not able to communicate their intention. One way of doing this is using traffic signals which only apply to human drivers \cite{Zhao2019}. The downside is that the nature of the intersection must remain similar to a traffic-light controlled one if non-communicating participants are going to be controlled by lights.

Recently a number of studies have addressed the extended intersection coordination of Connected and Autonomous Vehicles (CAVs) to the resolve the type of merging of diverging conflicts which occur and roundabouts. A centralized solution with an intersection manager minimizing delay and energy consumption is described in \cite{Zhao2018}. This shows that a high proportion of vehicles need to participate for significant benefits to be realized. 

A decentralized approach based on intent communication by way of virtual vehicles, can also be applied to roundabouts. In \cite{Debada2016}, reactive heuristics are shown to lead to poor performance compared to a model predictive control approach. The virtual vehicle concept allows common lane based heuristics such as car following to be extended to resolve conflicts in  \cite{Debada2018}. 

Another approach presented in \cite{Liu2018} is a decentralized as a solution to the global problem of minimizing the delay. Proofs of completeness and optimality of the aggregate problem are given, making this technique very impressive. It is not shown to be applicable to roundabouts in any of the numerical examples, although the incorporation of optimal trajectory planning by the low level controller to execute merging makes it a good example of the combination of path planning and intersection management. This collision constraints are based on a conflict zone rather than conflict points as in \cite{Levin2017}. The space inefficiency for multiple lanes is addressed by using multiple zones, one for each pair of lanes. The use of simulataneous path optimization might be expected to increase computational complexity and thereby reduce the number of vehicles with can be routed, however an attached video showing many vehicles interacting for about 10 minutes seems to refute this. It seem the ordering problem is resolved in a decentralized way based on game theory and the game `Chicken.' Solving large integer problems is known to be NP hard and may give this approach the edge over the mixed integer optimization used in \cite{Levin2017}, in terms of how many vehicles they can control before running into execution time limits. 