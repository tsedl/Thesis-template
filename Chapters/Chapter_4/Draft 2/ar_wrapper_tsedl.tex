%%%%%%%%%%%%%%%%%%%%%%% file template.tex %%%%%%%%%%%%%%%%%%%%%%%%%
%
% This is a general template file for the LaTeX package SVJour3
% for Springer journals.          Springer Heidelberg 2010/09/16
%
% Copy it to a new file with a new name and use it as the basis
% for your article. Delete % signs as needed.
%
% This template includes a few options for different layouts and
% content for various journals. Please consult a previous issue of
% your journal as needed.
%
%%%%%%%%%%%%%%%%%%%%%%%%%%%%%%%%%%%%%%%%%%%%%%%%%%%%%%%%%%%%%%%%%%%
%
% First comes an example EPS file -- just ignore it and
% proceed on the \documentclass line
% your LaTeX will extract the file if required
\begin{filecontents*}{example.eps}
%!PS-Adobe-3.0 EPSF-3.0
%%BoundingBox: 19 19 221 221
%%CreationDate: Mon Sep 29 1997
%%Creator: programmed by hand (JK)
%%EndComments
gsave
newpath
  20 20 moveto
  20 220 lineto
  220 220 lineto
  220 20 lineto
closepath
2 setlinewidth
gsave
  .4 setgray fill
grestore
stroke
grestore
\end{filecontents*}
%
\RequirePackage{fix-cm}
%
%\documentclass{svjour3}                     % onecolumn (standard format)
%\documentclass[smallcondensed]{svjour3}     % onecolumn (ditto)
%\documentclass[smallextended]{svjour3}       % onecolumn (second format)
\documentclass[twocolumn]{svjour3}          % twocolumn
%
\smartqed  % flush right qed marks, e.g. at end of proof
%
\usepackage{graphicx}
\graphicspath{{../IntersectionCore/}{../IntersectionCore/method_fig/}}
%
% \usepackage{mathptmx}      % use Times fonts if available on your TeX system
%
% insert here the call for the packages your document requires
%\usepackage{latexsym}
% etc.
%\usepackage{natbib} 
\usepackage[numbers]{natbib} % for numbered citations [1]
\usepackage{bm}
\usepackage{amsmath} % might be some conflict with mathpmtx
%
% please place your own definitions here and don't use \def but
% \newcommand{}{}
%
% Insert the name of "your journal" with
 \journalname{Springer Autonomous Robots}
%

\begin{document}

\title{Intersection Control for Automated Vehicles }
\thanks{ This research was made possible thanks to the financial support of a Full-time EPSRC Doctoral Training Partnership Studentship Institute for Transpoirt Studies and also thanks to CASE partner Guidance Automation Limited. 
%Grants or other notes
%about the article that should go on the front page should be
%placed here. General acknowledgments should be placed at the end of the article.}
}
%\subtitle{Do you have a subtitle?\\ If so, write it here}

%\titlerunning{Short form of title}        % if too long for running head

\author{Edward Derek Lambert         \and
        Richard Romano  \and David Watling %etc.
}

\authorrunning{ED Lambert \and R Romano \and DP Watling} % if too long for running head

\institute{E.D. Lambert \and R. Romano \and D.P.Watling \at
              Institute for Transport Studies, University of Leeds, 35- 41 Universiy Road, LS2 4JT  \\
              \email{tsedl@leeds.ac.uk}
		\email{R.Romano@leeds.ac.uk} 
		\email{D.P.Watling@its.leeds.ac.uk}            %  \\
%             \emph{Present address:} of F. Author  %  if needed
}

\date{Received: date / Accepted: date}
% The correct dates will be entered by the editor

\twocolumn[
  \begin{@twocolumnfalse}
\maketitle
\begin{abstract}
The contribution of the current work is to investigate the performance of certain per-intersection controller designs when the one-vehicle-per-segment assumption is relaxed. Particular attention has been paid to collision risks to ensure safe distances are always maintained over the simulation runs. A new waypoint interface makes it possible to simplify the constraints and find optimal approach speeds with a linear program, which is described in detail. Due to the linearity of the constraints, the optimal speeds for a large number of vehicles can be computed in a short time with interior point methods, guaranteeing completeness and ensuring infeasible problems are detected immediately. A longitudinal speed controller is described which allows simulated vehicles with limited motor power and electrical losses to arrive at the waypoints close to the set time, so that safe behaviour is guaranteed as long as deviations from the control model remain within tolerance. Total Travel Time is shown to be comparable to the quadratic constraints method despite the fixed crossing order, and much improved compared to a semaphore approach which also guarantees safety. Include keywords, PACS and mathematical
subject classification numbers as needed.
\keywords{First keyword \and Second keyword \and More}
% \PACS{PACS code1 \and PACS code2 \and more}
% \subclass{MSC code1 \and MSC code2 \and more}
\end{abstract}

\end{@twocolumnfalse}
]
\section{Introduction}

Reservation-based intersection management for preventing collisions between autonomous vehicles at intersections by V2V communication has been the topic of numerous studies considering road traffic. A good review is \cite{Chen2016}. Another review focusing on optimization methods is \cite{Malikopoulos2018}. Some early studies utilized a First-Come-First-Served (FCFS) policy which was shown to outperform signal control in some situations \cite{Dresner2008}. Problematic cases where performance could be worse than signal control were identified by \cite{Levin2016}. To capture the potential capacity improvements other works have used convex optimization \cite{Dai2017}.
The problem can also be posed as a mixed-integer optimization considering both the approach speed the arrival order as in \cite{Levin2017}.  
In \cite{Digani2019}, Digani et al present a per-intersection controller which calculates segment speeds for all approaching vehicles to minimize the total crossing time. They show that the reduction in crossing delay is far more than the increase in computation time for a realistic six way intersection, compared to a state-of-the-art decentralized approach. Like many existing studies of AGV co-ordination it is assumed that only one AGV at a time may be present on each path segment.
%
%    
%    
%\section{Literature Review}
%...
\input{../IntersectionCore/ic_lit_review}
%\section method + first results
\input{../IntersectionCore/intersection_control}
%\section{} ....
  
%\begin{acknowledgements}
%If you'd like to thank anyone, place your comments here
%and remove the percent signs.
%\end{acknowledgements}


% Authors must disclose all relationships or interests that 
% could have direct or potential influence or impart bias on 
% the work: 
%
% \section*{Conflict of interest}
%
% The authors declare that they have no conflict of interest.


% BibTeX users please use one of
\bibliographystyle{spbasic}      % basic style, author-year citations - Figure out why citation author lists dont fit in the columns 
%\bibliographystyle{spmpsci}      % mathematics and physical sciences
%\bibliographystyle{spphys}       % APS-like style for physics
\bibliography{ ar_references, ../roundabouts}   % name your BibTeX data base

%% Non-BibTeX users please use
%\begin{thebibliography}{}
%%
%% and use \bibitem to create references. Consult the Instructions
%% for authors for reference list style.
%%
%\bibitem{RefJ}
%% Format for Journal Reference
%Author, Article title, Journal, Volume, page numbers (year)
%% Format for books
%\bibitem{RefB}
%Author, Book title, page numbers. Publisher, place (year)
%% etc
%\end{thebibliography}

\end{document}
% end of file template.tex

