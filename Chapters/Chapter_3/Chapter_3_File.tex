%=== Chapter Three ===
\chapter{Fitting Smooth Arcs to Polygon Regions - Comparative Results}

\section{Introduction}
To understand the advantages of using clothoid pieces to join two poses in a 2D obstacle field, it is informative to compare the output paths with an alternative curve type. One well known option is to use polynomial curves of  degree 3. 

\section{Methodology}
The paths are compared on two simulated environments. One is dimensioned for a small AGV with turning radius 0.5m, as might be used to deliver small items in a flexible manufacturing environment. The unexpected obstacle completely blocks the path to the right as in Figure \ref{fig:env1}. The second could represent a fork lift type AGV with a larger turning radius of 2m, which is collecting standard size (1m x1.2m) pallets from a storage area. See Figure \ref{fig:env1}. The pallets are placed by human drivers, and a reliable system fo detecting their position and orientation is assumed to be in place. The AGV must manoeuvrer to the pose of a target pallet, without colliding with the others. Lower curvature and sharpness allow the AGV to traverse the path faster without compromising load stability.

\section{Algorithm}
Both methods decompose the problem into topology followed by curve fitting. The topology problem is a posed as a directed graph with weighted edges. There is a node for every intersection of the boundary between two regions. Nodes within the same region are fully connected. The weight of each edge corresponds to the euclidean distance between the two nodes. The A* Algorithm is used to search for the set of edges which give the minimum sum of weights between any start and end pose.

The sequence of edges is then used to populate the matrix $\bm{H} ^{(R \times P)} \in [0,1]$. This is a binary matrix containing $R$ columns, one for each of the $R$ polygonal regions which comprise the accessible space. Each row corresponds to one of the $P$ path pieces and contains a single non-zero element indicating the region to which is assigned. A path piece may be present in more than one region as the regions may be overlapping, but it must always remain completely inside its assigned region.

\subsection{Polynomial Method}
The problem specification calls for a path which changes smoothly in x, y, heading and curvature. This should start at a specified $x_s, y_s, \psi_s$ with zero curvature and end at $x_g, y_g, \psi_g$ with zero curvature.

\begin{equation}
\begin{array}{c}
x(t) = a + bt + ct^2 + dt^3 \\
y(t) = e + ft + gt^2 + ht^3 \\
\end{array}
\label{eq:spline}
\end{equation}

There is a unique solution for a cubic spline of with fixed (x, y, heading) at the start and goal, passing through fixed $x, y$ positions numbering $n$. A cubic spline defined by Equation \ref{eq:spline} has eight free parameters per segment. To give a unique solution, eight constraints are must be found for each segment. Passing through the $n$ waypoints at the end of each segment gives two, one for the $x$ coordinate and one for $y$. Enforcing continuity of position between the end of each segment and the next leads to two more. Four more can be determined from continuity in the first and second derivative of position for a total of eight. 

However, only four constraints are needed at the start and end of the spline. Fixing the final position and heading, the acceleration must be left free. Stacking the parameters into a vector 
\begin{equation}
\bm{p}_i = [a_i, b_i, c_i, d_i, e_i, f_i, g_i]^T
\end{equation} 
and 
\begin{equation}
\bm{p} = [\bm{p}_1,\cdots,\bm{p}_i, \cdots, \bm{p}_{n-1}]^T
\end{equation} 
leads to the system of linear equations is given in Equation \ref{eq:assembled}. 

\begin{equation}
[\bm{A}|\bm{b}_{x}] = \left[\begin{array}{cccccc|c}
\bm{A}_{0} & & & & \cdots & 0 & \bm{b}_{x0}\\
\bm{0} & \bm{A}_{1} & & & \cdots & 0 & \bm{b}_{x1}\\
\vdots & 	& \ddots &	& 	& \vdots	& \vdots \\
0 & & \cdots & \bm{A}_{i} &  \cdots & 0 & \bm{b}_{xi}\\
\vdots & 	& 	&	& \ddots  &		& \vdots \\
0 & \cdots &  &  & &  \bm{A}_{n} 			& \bm{b}_{xn}\\
\end{array}\right] 
\label{eq:assembled}
\end{equation}

Where
\begin{equation}
[\bm{A}_{0}|\bm{b}_{x0}] = \left[ \begin{array}{cccc|c}
1 & 0 & 0 & 0 & x_0\\
0 & 1 & 0 & 0 & \cos{\phi_0}\\
\end{array} \right] 
\label{eq:submatrix_0}
\end{equation}

\begin{equation}
[\bm{A}_{i}|\bm{b}_{xi}] = \left[\begin{array}{cccccccc|c}
1 & 1 & 1 & 1 & 0 & 0 & 0 & 0 & x_{i}\\
1 & 1 & 1 & 1 & -1 & 0 & 0 & 0 & 0\\
1 & 1 & 2 & 3 & 0 & -1 & 0 & 0 & 0\\
0 & 0 & 2 & 6 & 0 & 0 & -2 & 0 & 0
\end{array}\right]
\label{eq:submatrix_i}
\end{equation}

\begin{equation}
[\bm{A}_{n}|\bm{b}_{xn}] = \left[\begin{array}{cccccccc|c}
0 & 0 & 0 & 0 & 1 & 1 & 1 & 1 &x_n\\
0 & 1 & 0 & 0 & 0 & 1 & 2 & 3 & \cos{\psi_n}\\
\end{array}\right] 
\label{eq:submatrix_n}
\end{equation}

The $\bm{p}_x$ parameters to fit a set of $n$ waypoints can be found by computing $\bm{p}_x = \bm{A}^{-1}\bm{b}_x$. The $\bm{p}_y$ parameters can be found almost identically, by computing $\bm{p}_y=\bm{A}^{-1}\bm{b}_y$. Constraint vector $\bm{b}_y$ is instead constructed using the $y$ coordinates of the waypoints in $b_i$ and the $\sin$ of the start and end heading in $\bm{b}_{y0}$ and $\bm{b}_{yn}$.

\section{Optimisation}
The point to point solution through the region intersection points is quite a good solution. Continuous in $\dot{X}$ and $\ddot{X}$, it reaches the goal position and heading. It is excessively constrained by the requirement to meet the intermediate waypoints.
Using the optimisation in Equation \ref{eq:full_opt}  
\begin{equation}
\begin{array}{c}
\min \sum_{j=1}^N ||\frac{dP_j^3(t)}{dt^3}||^2 \\
\textrm{subject to} \\
P_j(1) = P_{j+1}(0) \\
\dot{P_j}(1) = \dot{P_j}(0) \\
\ddot{P_j}(1) = \ddot{P_j}(0) \\
P_1(0) = X_S \\ 
P_N(1) = X_G \\
\dot{P_1}(0) = \dot{X_S} \\
\dot{P_N}(1) = \dot{X_G} \\ 
\textrm{also subject to} \\
\bm{H}_{j,r} \Rightarrow X_r^{min} \leq P_j(t) \leq X_r^{max} 
\end{array}
\label{eq:full_opt}
\end{equation}

The implementation of the region constraint in Equation \ref{eq:full_opt} must ensure the entire arc remains inside the assigned region according to $\bm{H}$. For piecewise linear arcs this can be done by checking the containment of the start $P_0(0)$ and end $P_N(1)$. Got cubic splines it is more challenging. Deits and Tedrake \ref{Deits2015} describe an approach based on Sum of Squares, where a small Semidefinite Program is solved in the parameters of $P_j$, to avoid sampling at different $t$ values, which can lead to the path cutting corners and even passing through thin obstacles.

First we notice that the constraints on one segment from its axis aligned containing region described by $x_min, x_max, y_min, y_max$ can be written as vector inequality
\begin{equation}
 q(t) = \left[\begin{array}{c}
 x_{max} - a - bt - ct^2 - dt^3 \\ 
 a + bt+ ct^2 + dt^3 - x_{min} \\
 y_{max} - e - ft - gt^2 - ht^3 \\ 
 e + ft+ gt^2 + ht^3 - y_{min} \\
 \end{array}\right] \geq \bm{0} \forall t \in [0,1]
 \label{eq:qt}
 \end{equation}
. The condition in Equation \ref{eq:qt} can only hold if and only if it can be rewritten in the form
\begin{equation}
t\sigma_1(t) + (1-t)\sigma_2(t)
\end{equation}
. This leads to 
\begin{equation}
\sigma_1(t) = \left[\begin{array}{c}
	x_{max} - a- b- ct- dt^2 \\
	a + b+ ct+ dt^2 - x_{min} \\
	y_{max} - e - f - gt - ht^2 \\
	e + f+ gt+ ht^2 +-y_{min} \\
	\end{array}\right]
\end{equation}
and 
\begin{equation}
\sigma_2 = \left[\begin{array}{c}
x_{max} -a \\
a - x_{min} \\
y_{max} - e \\
e - y_{min} \\
\end{array}\right]
\end{equation}. 
The standard Sum of Squares approach calls for collecting the parameters of $\sigma(t)$ by the order of $t$. Matching coefficients against
\begin{equation}
\sigma_i = \beta_1 + \beta_2 t + \beta_3 t^2
\end{equation}, gives
\begin{equation}
\beta_1 = \left[\begin{array}{c}
x_{max} - a - b \\
a + b - x_{min} \\
y_{max} - e - f \\
e + f + y_{min} \\
x_{max} - a \\
a - x_{min} \\
y_{max} - e \\
e + y_{min} \\
\end{array}\right]
\end{equation}
and
\begin{equation}
\beta_2 = \left[\begin{array}{c}
-c \\
+c  \\
-c \\
+c \\
0\\
0 \\
0 \\
0 \\
\end{array}\right]
\end{equation}
and
\begin{equation}
\beta_3 = \left[\begin{array}{c}
-d \\
+d  \\
-d \\
+d \\
0\\
0 \\
0 \\
0 \\
\end{array}\right]
\end{equation}
. Where the parameters for $\sigma_2$ have been stacked after those from $\sigma_1$. 

In order for $\sigma_i$ to be a sum of squares, there are the following conditions on $\beta_1$, $\beta_2$, $\beta_3$:
\begin{equation}
\begin{array}{c}
4\beta_1\beta_3 - \beta_2^2 \geq 0
\beta_1, \beta_3 \geq 0
\end{array}
\end{equation}

 In the simple case where the regions are axis aligned, this immediately creates a problem as the first two equations have $\beta_3 = d$ and $\beta_3 = -d$. The only value of $d$ which will satisfy the sum of squares condition is zero. However the initial guess which joins the corner points, satisfies the region occupancy constraints with a non-zero $d$. 
 
 For this reason the constraints are enforced using 50 samples for each path piece. This leads to a small degree of corner cutting which must be included in the safety factor used on the vehicle body width used to inflate the obstacle. It also results in a large number of constraints as there are four for each sample. For this to be possible we must use a fixed spatial sampling length rather than a fixed number of samples per piece as path pieces can vary in length substantially which makes the degree of corner cutting variable.
 



\subsection{Clothoid Method}
Initialization with parameters which meet the continuity and obstacle constraints with clothoid pairs proved to be a challenging task. This is expected to improve the convergence time. To generate a set of parameters to meet continuity constraints for initialization of the region method is very similar to the interpolation problem addressed by Bertolazzi et al (2018) \cite{Bertolazzi2018}. The main difference is that the intermediate waypoint headings are free variables in the initialization, unlike in interpolation.

If the waypoint headings are fixed, and curvature at them to zero, the clothoid interpolation problem can be solved for each segment independently, as described by \cite{Gim2017a}. Different point to point methods could be used such as the bisection method \cite{Gim2017b}, geometry \cite{Vazquez-Mendez2016} and minimax sharpness \cite{Henrie2007}. Lacking a natural heading, a heuristic based on the maximum squared sum of $a^2 + b^2$ and requiring $a>0$ and $b<0$ where $a$ is the distance to the intersection point from $P_1$ and $b$ is the distance from $P_2$. The different sign is important for the direction of travel, to put the intersection in front of the first pose and behind the second.

The multiple shooting clothoid method uses additional parameters for the start pose of every segment. If this was initialized to zero, interior-point method reached convergence in about 70 seconds in environment 2. None of the more involved methods improved on this so we used the simple approach for further testing.  

\section{Convergence dependence on $w$}
In environment 1 both large and small values of $w$ led to poor convergence. One possible explanation is numerical conditioning being damaged by the weighting. A large $w$ results in a solution with small $\alpha$, while the length remains on the same order. Some matrix operations may return inaccurate results with different magnitude parameters.
\begin{table}
\begin{center}
	\begin{tabular}{ |c|c|c|c|c| }
		\hline
		 $w$ & $1\times10^{-3}$&  1 & 100 & 1000* \\ 
		 \hline
		 fevals &7929 & 10873 & 11986& 200,000* \\  
		 execution time (s)& 55 & 72 & 80 & 1345* \\
		 path length (m) & 7.071 & 11.694 & 36.038 & 5.695*\\
		 \hline  
	\end{tabular} \\
		 *Convergence Failure
\end{center}
\label{tab:w_dependence}
\caption{Convergence for different $w$ values for the multiple shooting approach, in pallet environment} 
\end{table}
To test the hypothesis that only method using dense matrices such as sqp would be affected, the effect on the interior point method (no dense matrix operation used in interior-point) is reported in Table 99. This shows that changing the weighting damages convergence significantly to the extent that no valid result is found even after 200000 function evaluations. There were no warnings printed by fmincon during this run. The value of feasibilty was positive of the order 1e-6 and decreasing very slowly, while the objective function was large of the order 1e6 and also decreasing very slowly (relative to its magnitude).

Contrary to expectation, reducing $w$ to emphasize the path length reduced the number of iterations to convergence. Table  \ref{tab:w_dependence} shows an inverse relationship between $w$ and the number of iterations. Sufficiently large $w$ makes the problem more difficult to solve as the path segments deflect further, making a linear approximation worse. This effect is indirect because the constraints use the precise nonlinear dynamics and the interior-point method does not solve a series of linear approximations to the constraints like sqp. Interior point should always converge if given an initial guess in the feasible reason, unless the problem in non-convex. The inital graph step intended to remove non-convex elements of the problem may lead to suboptimal paths as the segments deviate from straight lines. But this does not explain the increasing number of iterations. 

Curvature is more weakly linked to the parameters than the total length. The curvature objective is the squared sum of $2r$ parameters while the length objective is the squared sum of $4r$ parameters. Both of them leave numerous parameters to be coupled through the constraints as in total there are $9r$ parameters for the multiple shooting formulation.  

